\begin{document}
It's monday morning. You're sitting in your mathematics lecture. The professor asks you to solve the following equation:
	\begin{center}
		$5 + 2 = ?$
	\end{center}
Within a fraction of a second, you yell "It's seven!" and the professor lauds you. You lean back, thinking "Pfsh! That was easy."   You just unknowingly undervalued one of the most incredible tools that you have been blessed with: the human visual system. In each hemisphere of our brain, humans have a primary visual cortex (V1), containing 140 million neurons, with tens of billins of connections between them. There is an entire series of visual cortices doing progressively more complex image processing \cite{book}. \\ \\
We carry in our heads a supercomputer. We do not realize how good we are at making sense of our milieu because all the work is done unconsciously. Only when we attempt to write computer programs to emulate our neural processes is when we truly realize the complexity of the human mind. Nevertheless, the curiosity of scientists persisted and it was in 1958 that Frank Rosenblatt first introduced the idea of a concrete artificial mind. He named it the Perceptron, hoping that "it may eventually be able to learn, make decisions, and translate languages." \\ \\
It is the idea of the Perceptron that we explore in detail in this paper. We have divided it into five sections. First, we provide a gentle introduction to the concept of a perceptron. Second, we explore the mathematical foundations of a perceptron in depth to really understand what makes it work. 
\end{document}
